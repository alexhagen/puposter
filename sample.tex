\documentclass{puposter}

\begin{document}
\title{Progress on Tensioned Metastable Fluid Detectors for use in active
       interrogation schemes}
\author{Alex Hagen\tss{(1)}, Brian Archambault\tss{(2)},
        Anthony Sansone\tss{(1)}, Rusi Taleyarkhan\tss{(1),(2)}}
\affiliation{(1) Purdue University, (2) Sagamore Adams Laboratories, LLC.}
\conference{DNDO ARI, July 2016, Atlanta, GA}
\renewcommand{\today}{July 3, 2016}
\maketitle%
\hspace*{-1cm}%
\stepcounter{section}
\begin{pucol}{7}{12}
\def\numheadings{2}
\begin{pucell}{7}{6}{Tensioned Metastable Fluid Detectors}{}
    \begin{pucol}{4}{6}
    \begin{pucell}{4}{3}{}{}
        \begin{itemize}
            \item Current neutron detectors (such as He-3 and BF3 detectors) are limited by their gamma background, limited sensitivity, and the rising cost of the tube gas (in the case of He-3 tubes).
            \item TMFDs offer transformational benefits: 100\% gamma/beta blindness even in > 100R/h background; High intrinsic efficiency  (to 80\% neutrons; to ~100\% alpha/fission recoils); Directionality/Imaging/Multiplicity of SNMs in  a single portable detector; Simplicity/Lower costs
            \item Fluids may be placed in tension, much as solids. This tension manifests itself as negative (sub-zero) pressures.
            \item When a fluid is placed in sufficient tension, neutron recoil events can deposit enough energy to create a cavitation event.
            \item These cavitation events, if reaching critical size, can increase to macroscopic size, which, upon implosion, create audible implosion traces.
            \item The macroscopic cavitation events, paired with audible implosion events provide the ability to see and hear neutron detection events.
        \end{itemize}
    \end{pucell} \\
    \vspace*{-2.25in}
    \begin{pucell}{4}{3}{}{}
        \vspace*{-4cm}
            \posterfigure{img/tmfd_cartoons.png}{Diagram of two different TMFDs, CTMFD and ATMFD}{fig:tmfds}
    \end{pucell}
    \end{pucol}%
    \hspace*{\fill}
    \begin{pucol}{3}{4}
        \begin{pucell}{3}{1}{}{}
            \posterfigure{img/mfd_cartoon.png}{Schematic of metastable fluid detection mechanism}{fig:mfd}
        \end{pucell} \\
        \vspace*{\fill}
        \begin{pucell}{3}{3}{}{}
            \posterfigure{img/eff_vs_bulb_size.pgf}{Efficiency with size increase of CTMFD Bulbs}{fig:ctmfd-eff}
        \end{pucell} \\
        \begin{pucell}{3}{2}{}{}
            \vspace{-1cm}
            \begin{table}[H]
            \caption{Summary of highest measured efficiency for various sizes of CTMFD}
                \rowcolors{2}{grey20}{white}
                \def\arraystretch{2.5}
                \hspace*{-1cm}
                \begin{tabular}{>{\centering}m{0.25\textwidth}>{\centering}m{0.37\textwidth}>{\centering}m{0.37\textwidth}}
                \rowcolor{newgold}
                \champion{CTMFD SV ($\mathrm{cm^{3}}$)} & \champion{Highest Measured Intrinsic Efficiency} & \champion{Theoretical Maximum Intrinsic Efficiency}\tabularnewline
                4 & $13.2\pm1.9\%$ & $15\%$\tabularnewline
                15 & $27.4\pm4.5\%$ & $35\%$\tabularnewline
                40 & $56\pm8.6\%$ & $60\%$\tabularnewline
                \end{tabular}
            \end{table}
        \end{pucell}
    \end{pucol}
\end{pucell} \\
\vspace*{\fill}
\begin{pucell}{7}{6}{Threshold Energy Neutron Analysis}{}
  % CTMFD Row
  \begin{pucell}{3}{3}{Centrifugal TMFDs}{Efficiency and Rejection Experimentation}
    \begin{itemize}
      \item TENA uses a continuously driven $2.45\,MeV$ $\left(D,D\right)$ interrogation source
    \end{itemize}
    \begin{table}[H]
        \vspace*{-0.5cm}
        \caption{Rejection ratio and efficiency for $16\,cm^{3}$ SV and $40\,cm^{3}$ SV CTMFDs}
        \rowcolors{2}{grey20}{white}
        \def\arraystretch{1.0}
        \hspace*{0.5cm}
        \begin{tabular}{>{\centering}m{0.15\textwidth}>{\centering}m{0.25\textwidth}>{\centering}m{0.25\textwidth}>{\centering}m{0.25\textwidth}}
            \rowcolor{newgold}
            CTMFD SV ($cm^{3}$) & DD Rejection Ratio ($\eta_{Cf}:\eta_{DD}$) & $p_{neg}$ ($bar$) & $^{252}Cf$ Instrinsic Detecton efficiency $\eta_{Cf}$ ($E_{n}>2.5\,MeV$)\tabularnewline
            $16$ & $10^{3}:1$ & $3.2$ & $1.3\%$\tabularnewline
            $16$ & $10^{4}:1$ & $<2.8$ & --\tabularnewline
            $40$ & $10^{3}:1$ & $3.75$ & $4.0\%$\tabularnewline
            $40$ & $10^{4}:1$ & $3.4$ & $1.6\%$\tabularnewline
        \end{tabular}
    \end{table}
  \end{pucell}%
  \hspace*{\fill}
  \begin{pucell}{2}{3}{}{}
    \posterfigure{img/m16_eff.pgf}{Efficiency Ratio ($\eta_{Cf} : \eta_{DD}$) between $Cf$ and $\left(D,D\right)$ for $40\,cm^{3}$ bulbed CTMFD}{fig:mctmfd}
  \end{pucell}%
  \hspace*{\fill}
  \begin{pucell}{2}{3}{}{}
    \posterfigure{img/r5_eff.pgf}{Efficiency between $\left(D,D\right)$ and $Cf$ for $40\,cm^{3}$ bulbed CTMFD}{fig:mctmfd}
  \end{pucell} \\
  \vspace*{0.25in} \\
  % ATMFD Row
  \begin{pucell}{3}{3}{Economical Acoustic TMFDs}{Efficiency and Rejection Experimentation}
    \begin{itemize}
      \item Dead time in decreased compared to CTMFDs, but the negative pressure profile is more complex
      \item With an EATMFD, rejection ratios between $\left(D,D\right)$ and $\mathrm{Cf}$ were found to be at most $1:10^{2}$
      \item Using a DATMFD, rejection rates between $\mathrm{\left(D,D\right)}$ and $\mathrm{Cf}$ were found to be up to $1:10^{3}$
      \item The increase in rejection with the DATMFD compared to the EATMFD is hypothesized to be because of the simpler and flatter negative pressure field in the DATMFD
      \item The directional and $ns$ timing features of the DATMFD could also be used to veto counts from the interrogation source, which may provide higher rejection
    \end{itemize}
  \end{pucell}%
  \hspace*{\fill}
  \begin{pucell}{2}{3}{}{}
    \posterfigure{img/ef15_eff.pgf}{Efficiency between $\mathrm{\left(D,D\right)}$ and $\mathrm{Cf}$ for an Economical ATMFD system}{fig:eatmfd}
  \end{pucell}%
  \hspace*{\fill}
  \begin{pucell}{2}{3}{}{}
    \posterfigure{img/datmfd_eff.pgf}{Efficiency between $\mathrm{\left(D,D\right)}$ and $\mathrm{Cf}$ for an Directional ATMFD system}{fig:datmfd}
  \end{pucell} \\
  \vspace*{\fill}
\end{pucell}%
\end{pucol}%
\hspace*{1.5in}
\stepcounter{section}
\begin{pucol}{5}{12}
  \def\numheadings{2}
  \begin{pucell}{5}{8}{Photointerrogation}{}
    \begin{pucol}{2}{8}
      \begin{pucell}{2}{5}{Clinac 6Ex Interrogation}{Interrogation Results with $6\,MeV$ \scalebox{2.0}{$\gamma$}}
        \begin{itemize}
          \item TMFDs are blind to photons, which makes them ideally suited for photointerrogation environments
          \item The TMFDs have proven blindness in up to a $>100\,\mathrm{\frac{R}{hr}}$ $\mathrm{^{140}La}$ field ($1.6\,\mathrm{MeV}$ \scalebox{2.0}{$\gamma$})
          \item TMFDs have been tested in a photointerrogation environment using a $6\,\mathrm{MV}$ Varian CLINAC accelerator with a $\mathrm{W}$ target
          \item Under photointerrogation, a CTMFD at $5\,\mathrm{bar}$ was able to detect $630\,\mathrm{g}$ of natural uranium dioxide while remaining blind to photons
          \item At higher negative pressure, photoneutron interrogation from the $\left(\gamma,n\right)$ reaction on $\mathrm{^{2}H}$ contaminates the interrogation beam
          \item Using lower negative pressure and standoff was able to gate out contaminating photoneutrons, but schemes for shielding the detector or photoneutron creating materials are in development
          \item If photoneutron contamination can be reduced, the full efficiency of the CTMFD detector could be used, significantly decreasing the mass of Uranium that could be detected
        \end{itemize}
      \end{pucell} \\
      \vspace*{\fill}
      \begin{pucell}{2}{3}{}{}
        \posterfigure{img/clinac_setup_cartoon.pgf}{Setup of CTMFD Detector under $6\,MeV$ photon beamline}{fig:clinacsetup}
      \end{pucell}
    \end{pucol}%
    \begin{pucol}{3}{8}
      \begin{pucell}{3}{3}{}{}
        \posterfigure{img/photoneutron_detection.pgf}{Efficiency of detection of NU and Be targets in $6\,MeV\,\gamma$ beamline}{fig:clinacefficiency}
      \end{pucell} \\
      \vspace*{\fill}
      \begin{pucell}{3}{5}{}{}
        \posterfigure{img/setup_plot.pgf}{Photoneutron production in Clinac Environment}{fig:photoneutron}
      \end{pucell}
    \end{pucol}%
  \end{pucell}
  \vspace*{\fill} \\
  \begin{pucell}{5}{4}{Future Work}{}
    \begin{pucell}{2}{4}{DDAA}{Differential Die Away Analysis}
      \vspace*{\fill}
      \begin{itemize}
        \item Differential Die Away Analysis uses a pulsed $14.1\,MeV$ $\left(D,T\right)$ interrogation source
        \item DDAA type schemes are possible with TMFDs. The Acoustically Tensioned Metastable Fluid Detector could be phase locked to be insensitive during the interrogation pulse, thus only detecting on a die away signal.
        \item Phase locking will guard against detection of the interrogation beam, to allow operation at high power to maximize detection efficiency.
        \item The accurate event timing ($<\mathrm{ns}$) will allow for further rejection of interrogation beam by veto.
      \end{itemize}
      \vspace*{\fill}
    \end{pucell}%
    \hspace*{\fill}
    \begin{pucell}{3}{4}{}{}
      \posterfigure{img/atmfd_p.pgf}{Differential Die Away Schematic for Using DATMFDs}{fig:ddaa}
    \end{pucell}
  \end{pucell}%
\end{pucol}
\begin{tikzpicture}[remember picture, overlay]
    \node[inner sep=0, outer sep=0, text width=22in, text height=1.5in,
          anchor=south west] (ack)
          at ($(current page.south west) + (1.5in, 1.0in)$)
          {
          References available upon request.\\
          \textbf{Acknowledgements:}\\
          Purdue University Faculty and Staff: Drs. J. Poulson and N. Rancilio
          (Purdue Veterinary Medicine Faculty), Dr. T. Grimes, J. Webster,
          A. Bakken (Graduate Researchers), N. Boyle, A. Early, S. Dissler,
          Purdue REM Staff (Radiological and Chemical Safety) \\
          Jefferson High School Community Outreach: J. Ruhl (Science Teacher),
          C. Mason, N. Mondero, S. Tan, C. Hainje, P. Stamper, R. Scott
          (Involved Students) \\
          Industrial Collaborators: Dr. B. Archambault,
          Sagamore Adams Laboratories (www.salabsllc.com)
          };
      \node[inner sep=0, outer sep=0, text width=22in, text height=1.5in,
            anchor=south east, align=right] (logos)
            at ($(current page.south east) + (-1.5in, 1.0in)$)
            {\includegraphics[scale=1.25]{img/logos.pdf}};
\end{tikzpicture}
\end{document}
